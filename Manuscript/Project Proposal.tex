\documentclass{article}
\usepackage[utf8]{inputenc}
\usepackage{amsmath}

\title{A Discrete-time Markov Process for Predicting Influenza Evolution.
}
\author{Emily Brunelli$^*$, Gunnar Johnson$^*$, Jonathan Mah$^*$}
\date{Math 381, Summer 2018}

\begin{document}

\begin{center}
    \textbf{PROJECT PROPOSAL}
\end{center}

\begin{enumerate}
    \item \textbf{The people on your team.}\\
    Emily Brunelli, Gunnar Johnson, Jonathan Mah
    \item \textbf{A project title.}\\
    A Discrete-time Markov Process for Predicting Influenza Evolution.
    \item \textbf{A clear statement of the goal of the project, and a clear explanation of the phenomenon to be modeled.}\\
    We hope to be able to predict viral evolution, as well as gain insight towards the ``evolutionary space" available.
    \item \textbf{Importance of the problem. Who will care about your solution? How did you come up with the project idea?}\\
    Certain viral diseases, like Influenza and HIV, currently lack an effective treatment due to their extreme rate and diversity of evolution. In hopes of one day curing these diseases, it is imperative to understand the underlying biological mechanisms which allow these diseases to escape the human immune response. One of our members works in a lab which investigates this type of problem, but uses experimental (wet-lab) data and maximum likelihood instead of discrete mathematics.
    \item \textbf{An idea of the methods you plan to use in modeling your problem.}\\
    Given input data consisting of strain-specific FASTA sequences, we will iterate through a discrete-time markov chain in an attempt to predict the rate and diversity of the indicated virus's evolution.
    \item \textbf{An idea of how you will gather data for your project. You must use real data whenever possible.}\\
    We will be downloading FASTA alignments from the NCBI Influenza Virus Database.
    \item \textbf{Provide a small example of your main problem. Translate it into a mathematical problem which you can solve.}\\
    A basic form of this problem involves 4-dimensional FASTA sequences. After being chronologically sorted, a ``sequence matrix" is constructed where each row represents one sequence. Difference vectors are calculated by column in order to calculate an index-specific probability matrix, denoted as $A_i$.
    \[
    A_i =
        \begin{bmatrix}
            \pi_{AA} & \pi_{AC} & \pi_{AG} & \pi_{AT} \\
            \pi_{CA} & \pi_{CC} & \pi_{CG} & \pi_{CT} \\
            \pi_{GA} & \pi_{GC} & \pi_{GG} & \pi_{GT} \\
            \pi_{TA} & \pi_{TC} & \pi_{TG} & \pi_{TT}
        \end{bmatrix}
    \]
    where $\pi_{jk}$ represents the probability of a site $i$ evolving from $j$ to $k$. We plan to expand upon this problem by translating DNA into Amino acids, which results in FASTA sequences with 20 dimensions instead of 4, and consequently, a $20 \times 20$ transition matrix. Additionally, we were interested in possibly computing the variation of a sites evolution, and mapping out specific sites which evolve faster than expected.
    \item \textbf{A reference related to your topic (research articles, news articles, textbooks, previous work on the subject).}\\ This article
    \\ http://bedford.io/pdfs/papers/morris-predictive-modeling.pdf \\ titled ``Predictive Modeling of Influenza Shows the Promise of Applied Evolutionary Biology" is a good review of current efforts to predict Influenza evolution for vaccine design.
    \item \textbf{A reference related to your proposed method of solution.}\\ This article \\ https://academic.oup.com/mbe/article/22/5/1208/1066893 \\
    titled ``Not So Different After All: A Comparison of Methods for Detecting Amino Acid Sites Under Selection" gives a great review of different methods for tackling this type of problem. Most methods start from the basic $4 \times 4$ Markov chain and then expand upon it. For example, some papers introduce external weighting factors like protein structure, and some papers define a site's expected rate of evolution in order to detect abnormalities.
\end{enumerate}
\newpage
\maketitle
\footnotesize{$^*$Denotes equal contribution to work}\\

\section{Abstract}
\section{Introduction}
\section{Assumption}
\section{Mathematical Model}
\section{Solution of the Mathematical Problem}
\section{Testing}
\section{Results}
\section{Improvements}
\section{Conclusions}
\section{Acknowledgements}
\section{References}

\end{document}
